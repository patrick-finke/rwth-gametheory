\documentclass[11pt]{beamer}
\usepackage[utf8]{inputenc}
\usepackage[ngerman]{babel}
\usepackage{amsmath}
\usepackage{amsfonts}
\usepackage{amssymb}
\usepackage{tcolorbox}
\usepackage{mathtools} % \coloneqq
\usepackage{multirow} % better tables, www.tablesgenerator.com
\usetheme{rwth}

% Define RWTH style colors

%%%%%%%%%%%%%%%%%%%%%%%%%%%%%%%%%%%%%%%%%%
% For HKS colors use the spotcolor package
% which predefines a number of such colors
% Main blue color is HKS 44 K
%%%%%%%%%%%%%%%%%%%%%%%%%%%%%%%%%%%%%%%%%%

%%%%%%%%%%%%%%%%%%%%%%%%%%%%%%%%%%%%%%%%%%
% RGB colors used for screen
%%%%%%%%%%%%%%%%%%%%%%%%%%%%%%%%%%%%%%%%%%
% The main blue color, 100 and 50 are the
% most commonly used ones
% e.g. in the logo
\xdefinecolor{rwth}   {RGB}{  0  84 159}
\xdefinecolor{rwth-75}{RGB}{ 64 127 183}
\xdefinecolor{rwth-50}{RGB}{142 186 229}
\xdefinecolor{rwth-25}{RGB}{199 221 242}
\xdefinecolor{rwth-10}{RGB}{232 241 250}

% All the other colors
% Secondary colors
\xdefinecolor{black}   {RGB}{  0   0   0}
\xdefinecolor{black-75}{RGB}{100 101 103}
\xdefinecolor{black-50}{RGB}{156 158 159}
\xdefinecolor{black-25}{RGB}{207 209 210}
\xdefinecolor{black-10}{RGB}{236 237 237}

\xdefinecolor{magenta}   {RGB}{227   0 102}
\xdefinecolor{magenta-75}{RGB}{233  96 136}
\xdefinecolor{magenta-50}{RGB}{241 158 177}
\xdefinecolor{magenta-25}{RGB}{249 210 218}
\xdefinecolor{magenta-10}{RGB}{253 238 240}

\xdefinecolor{yellow}   {RGB}{255 237   0}
\xdefinecolor{yellow-75}{RGB}{255 240  85}
\xdefinecolor{yellow-50}{RGB}{255 245 155}
\xdefinecolor{yellow-25}{RGB}{255 250 209}
\xdefinecolor{yellow-10}{RGB}{255 253 238}

% The extended color spectrum
\xdefinecolor{petrol}   {RGB}{  0  97 101}
\xdefinecolor{petrol-75}{RGB}{ 45 127 131}
\xdefinecolor{petrol-50}{RGB}{125 164 167}
\xdefinecolor{petrol-25}{RGB}{191 208 209}
\xdefinecolor{petrol-10}{RGB}{230 236 236}

\xdefinecolor{turkis}   {RGB}{  0 152 161}
\xdefinecolor{turkis-75}{RGB}{  0 177 183}
\xdefinecolor{turkis-50}{RGB}{137 204 207}
\xdefinecolor{turkis-25}{RGB}{202 231 231}
\xdefinecolor{turkis-10}{RGB}{235 246 246}

\xdefinecolor{grun}   {RGB}{ 87 171  39}
\xdefinecolor{grun-75}{RGB}{141 192  96}
\xdefinecolor{grun-50}{RGB}{184 214 152}
\xdefinecolor{grun-25}{RGB}{221 235 206}
\xdefinecolor{grun-10}{RGB}{242 247 236}

\xdefinecolor{maigrun}   {RGB}{189 205   0}
\xdefinecolor{maigrun-75}{RGB}{208 217  92}
\xdefinecolor{maigrun-50}{RGB}{224 230 154}
\xdefinecolor{maigrun-25}{RGB}{240 243 208}
\xdefinecolor{maigrun-10}{RGB}{249 250 237}

\xdefinecolor{orange}   {RGB}{246 168   0}
\xdefinecolor{orange-75}{RGB}{250 190  80}
\xdefinecolor{orange-50}{RGB}{253 212 143}
\xdefinecolor{orange-25}{RGB}{254 234 201}
\xdefinecolor{orange-10}{RGB}{255 247 234}

\xdefinecolor{rot}   {RGB}{204   7  30}
\xdefinecolor{rot-75}{RGB}{216  92  65}
\xdefinecolor{rot-50}{RGB}{230 150 121}
\xdefinecolor{rot-25}{RGB}{243 205 187}
\xdefinecolor{rot-10}{RGB}{250 235 227}

\xdefinecolor{bordeaux}   {RGB}{161  16  53}
\xdefinecolor{bordeaux-75}{RGB}{182  82  86}
\xdefinecolor{bordeaux-50}{RGB}{205 139 135}
\xdefinecolor{bordeaux-25}{RGB}{229 197 192}
\xdefinecolor{bordeaux-10}{RGB}{245 232 229}

\xdefinecolor{violett}   {RGB}{ 97  33  88}
\xdefinecolor{violett-75}{RGB}{131  78 117}
\xdefinecolor{violett-50}{RGB}{168 133 158}
\xdefinecolor{violett-25}{RGB}{210 192 205}
\xdefinecolor{violett-10}{RGB}{237 229 234}

\xdefinecolor{lila}   {RGB}{122 111 172}
\xdefinecolor{lila-75}{RGB}{155 145 193}
\xdefinecolor{lila-50}{RGB}{188 181 215}
\xdefinecolor{lila-25}{RGB}{222 218 235}
\xdefinecolor{lila-10}{RGB}{242 240 247}


\def\remark#1{\textcolor{red}{#1}}

\def\R{\mathbb{R}}

\AtBeginSection{\frame[noframenumbering,plain]{
		\frametitle{}
		\tableofcontents[currentsection]
}}

\begin{document}
    \title[Seminar Optimization]{Seminar Optimization}
    \author[Patrick]{Patrick Finke}
    \email{patrick.finke@rwth-aachen.de}
	\subtitle{Introduction to \textbf{Game Theory and Static Games}\\based on Bressan, \emph{Noncooperative Differential Games. A Tutorial, 2010}}
	%\logo{}
	%\institute{}
    \date[]{January 29th, 2019}
	%\subject{}
	%\setbeamercovered{transparent}
	%\setbeamertemplate{navigation symbols}{}
	\frame[plain]{\maketitle}

    \begin{frame}
        \tableofcontents
    \end{frame}
	
    \section{Introduction}
    \begin{frame}{\secname}
		\begin{tcolorbox}[colback=rwth-10,colframe=rwth,title=Static Game (for two players), sharp corners]
            \begin{itemize}
                \item two sets of strategies $A$ and $B$
                \item payoff functions $\Phi^A, \Phi^B: A \times B \rightarrow \R$
                \item goal of player A: $\max_{a \in A} \Phi^A(a, b)$
                \item goal of player B: $\max_{b \in B} \Phi^B(a, b)$
            \end{itemize}
        \end{tcolorbox}

        \vfill

		\begin{tcolorbox}[colback=rwth-10,colframe=rwth,title=Assumption, sharp corners]
			\begin{equation}
				\label{eq-assumption}
				\tag{A}
            	A, B\textrm{ compact metric spaces, }\Phi^A, \Phi^B\textrm{ continuous}
            \end{equation}
        \end{tcolorbox}
	\end{frame}

    \begin{frame}{\secname}
    \textcolor{rwth}{Example:} Prisoner's dilemma
    
    \vfill
    
    \begin{figure}
    	\centering
		\includegraphics[width=0.7\textwidth]{pics/prisoner.jpg}
	\end{figure}

	\vfill
	
	\textcolor{rwth}{Question:} Which pair of strategies is optimal?
    \end{frame}

    \section{Solution Concepts}
    % Pareto
    \begin{frame}{\secname}
		\begin{tcolorbox}[colback=rwth-10,colframe=rwth,title=Pareto optimality, sharp corners]
            $(a^*, b^*)$ is Pareto optimal if there exist no $(a, b) \in A \times B$ such that
            $$\Phi^A(a, b) > \Phi^A(a^*, b^*) \quad \text{and} \quad \Phi^B(a, b) \geq \Phi^B(a^*, b^*)$$
            \center{or}
            $$\Phi^B(a, b) > \Phi^B(a^*, b^*) \quad \text{and} \quad \Phi^A(a, b) \geq \Phi^A(a^*, b^*)$$
        \end{tcolorbox}
    
        \vfill

        It is not possible to strictly increase the payoff of one player without strictly decreasing the payoff of the other.
        
       	\vfill

       	\textcolor{rwth}{Note:} favourable from a social standpoint
    \end{frame}

    \begin{frame}{\secname}
	\textcolor{rwth}{Example:} prisoner's dilemma
	
	\vfill
	
	\begin{figure}
		\centering
		\includegraphics[width=0.7\textwidth]{pics/prisoner_pareto.jpg}
	\end{figure}
	
	\center{Pareto optimal solutions in red.}
	\end{frame}

    % Stackelberg
    \begin{frame}{\secname}

        \textcolor{rwth}{Idea:}
        \begin{itemize}
            \item A leader, B follower
            \item A announces $a \in A$
            \item B chooses $b^* \in R^B(a)$ say $b^* = \beta(a)$
            \item goal of player A is $\max_{a \in A} \Phi^A(a, \beta(a))$
        \end{itemize}

        \vfill

		\begin{tcolorbox}[colback=rwth-10,colframe=rwth,title=Stackelberg equilibrium, sharp corners]
            $(a_S, b_S)$ is Stackelberg equilibrium if
            \begin{itemize}
                \item $b_S \in R^B(a_S)$
                \item $\Phi^A(a, b) \leq \Phi^A(a_S, b_S) \quad \forall (a, b) \textrm{ with } b \in R^B(a)$
            \end{itemize}
        \end{tcolorbox}
    
    	\vfill
    	
    	\textcolor{rwth}{Note:} Models asymmetry of information
    
    	\vfill
    	
    	(best reply map: $R^B(a) = \{ b \in B \mid \Phi^B(a,b) = \max_{\omega \in B} \Phi^B(a, \omega) \}$)
    \end{frame}

    \begin{frame}{\secname}
		\begin{tcolorbox}[colback=rwth-10,colframe=rwth,title=Nash equilibrium, sharp corners]
            $(a^*, b^*)$ is Nash equilibrium if for every $a \in A, b \in B$
            $$\Phi^A(a, b^*) \leq \Phi^A(a^*, b^*)$$
            \center{and}
            $$\Phi^B(a^*, b) \leq \Phi^B(a^*, b^*)$$
        \end{tcolorbox}

        \vfill

        No player can increase his payoff by changing his strategy as long as the other player sticks to the equilibrium strategy.
        
        \vfill
        
        \textcolor{rwth}{Note:} symmetric situation, no means to cooperate, do not share information
    \end{frame}

    \begin{frame}{\secname}
	\textcolor{rwth}{Example:} prisoner's dilemma
	
	\vfill
	
	\begin{figure}
		\centering
		\includegraphics[width=0.7\textwidth]{pics/prisoner_nash.jpg}
	\end{figure}
	
	\center{Nash equilibria in red.}
	\end{frame}

	\begin{frame}{\secname}
		\textcolor{rwth}{Some properties of Nash equilibria are:}
		\begin{enumerate}
			\item Nash equilibria may not exist,
			\item Nash equilibria need not be unique,
			\item different Nash equilibria can yield different payoffs,
			\item a Nash equilibrium may not be Pareto optimal.
		\end{enumerate}
	\end{frame}

    \section{Existence of Nash Equilibria}
    \begin{frame}{\secname}
		\begin{tcolorbox}[colback=rwth-10,colframe=rwth,title=Theorem, sharp corners]
            Assume $A, B \subset \R^n$ compact, convex; $\Phi^A, \Phi^B$ continuous and
            $$a \mapsto \Phi^A(a,b)\textrm{ is a concave function of }a\textrm{, for all }b \in B$$
            $$b \mapsto \Phi^B(a,b)\textrm{ is a concave function of }b\textrm{, for all }a \in A$$
            Then the non-cooperative game admits a Nash equilibrium.
        \end{tcolorbox}
	\end{frame}

	\begin{frame}{\secname}
		\begin{minipage}{.15\textwidth}
			\textcolor{rwth}{Recall:} 
		\end{minipage}%
		\begin{minipage}{.7\textwidth}
			$R^A(b) = \{ a \in A \mid \Phi^A(a, b) = \max_{\omega \in A} \Phi^A(\omega, b)\}$\\
			$R^B(a) = \{ b \in B \mid \Phi^B(a, b) = \max_{\omega \in B} \Phi^B(a, \omega)\}$
		\end{minipage}
	
		\vfill
		
		\begin{tcolorbox}[colback=rwth-10,colframe=rwth,title=Nash equilibrium as fixed point, sharp corners]
			$(a^*, b^*)$ is Nash equilibrium iff
			$$a^* \in R^A(b^*)\textrm{ and }b^* \in R^B(a^*).$$
		\end{tcolorbox}
	
		\vfill
	
		Since
		\begin{align*}
			& \Phi^A(a, b^*) \leq \Phi^A(a^*, b^*) \ \forall a \in A\\
			\Leftrightarrow {} & {} \Phi^A(a^*, b^*) = \max_{a \in A} \Phi^A(a, b^*)\\
			\Leftrightarrow {} & {} a^* \in R^A(b^*) 
		\end{align*}
	\end{frame}


	\begin{frame}{\secname}
		\begin{tcolorbox}[colback=rwth-10,colframe=rwth,title=upper semi-continuous, sharp corners]
			$Y$ compact, $F: X \rightrightarrows Y$ a multifunction with nonempty, compact values. Then the following are equivalent:
			\begin{itemize}
				\item $F$ is upper semicontinuous (i.e. $\forall x \in X \, \forall \epsilon > 0 \, \exists \delta = \delta(\epsilon) > 0$ such that $F(x') \subseteq B_{\epsilon}(F(x))$ for all $d(x', x) < \delta$)
				\item $graph(F) = \{ (x,y) \mid y \in F(x) \}$ is closed
				%\item $F, G$ upper semicontinuous implies $(F \times G): (x,y) \mapsto F(y) \times G(x)$ also upper semicontinuous
			\end{itemize}
			
		\end{tcolorbox}
		\begin{tcolorbox}[colback=rwth-10,colframe=rwth,title=Kakutani, sharp corners]
            Let $K \subset \R^n$ compact, convex. Let $F: K \rightrightarrows \R^n$ be an upper semicontinuous multifunction with compact, convex values such that $F(x) \subset K$ for all $x \in K$. Then there exists $\bar{x} \in K$ such that
            \center{$\bar{x} \in F(\bar{x}).$}
        \end{tcolorbox}
    \end{frame}

    \section{Randomized Strategies}
    
    \begin{frame}{\secname}
    	\begin{tcolorbox}[colback=rwth-10,colframe=rwth,title=Randomized Strategies, sharp corners]
    		\begin{itemize}
    			\item strategy for player A is probability distribution $\mu \in P(A)$
    			\item strategy for player B is probability distribution $\nu \in P(B)$
    			\item payoff functions
    			$$\tilde{\Phi}^A(\mu, \nu) = \int_{A \times B} \Phi^A(a,b) \, d(\mu, \nu)$$
    			$$\tilde{\Phi}^B(\mu, \nu) = \int_{A \times B} \Phi^B(a,b) \, d(\mu, \nu)$$
    		\end{itemize}
        \end{tcolorbox}
    	
    	\vfill
    	
    	\textcolor{rwth}{Note:} $\tilde{\Phi}^A, \tilde{\Phi}^B$ are expected values of the payoffs, if $\mu$ and $\nu$ are chosen independently
    	
    	\textcolor{rwth}{Note:}  $\{\textrm{pure strategies}\}\subseteq \{\textrm{randomized strategies}\}$
    \end{frame}

	\begin{frame}{\secname}
	    \begin{tcolorbox}[colback=rwth-10,colframe=rwth,title=existence of Nash equilibria for randomized strategies, sharp corners]
    		Let \eqref{eq-assumption} hold. There exist $\mu^* \in P(A)$ and $\nu^* \in P(B)$ such that
    		$$\tilde{\Phi}^A(\mu, \nu^*) \leq \tilde{\Phi}^A(\mu^*, \nu^*)\textrm{ for all } \mu \in P(A)$$
    		\center{and}
    		$$\tilde{\Phi}^B(\mu^*, \nu) \leq \tilde{\Phi}^B(\mu^*, \nu^*)\textrm{ for all } \nu \in P(B)$$
    	\end{tcolorbox}
	\end{frame}

    \section{Zero-Sum Games}
    
    \begin{frame}{\secname}
  		\begin{tcolorbox}[colback=rwth-10,colframe=rwth,title=Zero-Sum Game, sharp corners]
   			\begin{itemize}
   				\item consider again static game: $A$, $B$, $\Phi^A$, $\Phi^B$
   				\item spacial case $\Phi^A + \Phi^B = 0$
   				\item single payoff function sufficient
				$$\Phi \coloneqq \Phi^A = - \Phi^B$$
				\item goal of player A is $\max_{a \in A} \Phi(a, b)$\\
				goal of player B is $\min_{b \in B} \Phi(a, b)$
   			\end{itemize}
   		\end{tcolorbox}
   	
   		\vfill
   	
  		\begin{tcolorbox}[colback=rwth-10,colframe=rwth,title=Assumption, sharp corners]
			\begin{equation*}
				\label{eq-assumption-zerosum}
				\tag{A'}
				A, B\textrm{ compact metric spaces, }\Phi\textrm{ continuous}
			\end{equation*}
		\end{tcolorbox}
    \end{frame}

	\begin{frame}{\secname}
		Suppose advantage of information:
		\begin{itemize}
			\item player B chooses $b \in B$
			\item player A chooses $\alpha(b) \in A$ such that
			$$\Phi(\alpha(b), b) = \max_{a \in A} \Phi(a,b)$$
			\item goal of player B is $\min_{b \in B} \Phi(\alpha(b), b)$
		\end{itemize}
		This yields:
		$$V^- \coloneqq \max_{a \in A} \min_{b \in B} \Phi(a, b) \leq \min_{b \in B} \max_{a \in A} \Phi(a,b) \eqqcolon V^+$$
		
		\vfill
		
		\begin{tcolorbox}[colback=rwth-10,colframe=rwth,title=Value of the Game, sharp corners]
			If $V^- = V^+ \eqqcolon V$, then $V$ is called value of the game.
		\end{tcolorbox}
	\end{frame}

	%\begin{frame}{\secname}
	%	\remark{DELETE IF TOO LONG}
	%	\begin{tcolorbox}[colback=rwth-10,colframe=rwth,title=saddle point, sharp corners]
	%		$(a^*, b^*) \in A \times B$ such that
	%		$$\min_{b \in B} \Phi(a^*, b) = \Phi(a^*, b^*) = \max_{a \in A} \Phi(a, b^*)$$
	%	\end{tcolorbox}
	%
	%	\vfill
	%
	%	\textcolor{rwth}{Note:} For a zero-sum game each saddle point is a Nash equilibrium and vice versa, e.g.
	%	\begin{align*}
	%		& {} \Phi(a, b^*) \leq \Phi(a^*, b^*)\\
	%		\Leftrightarrow {} & {} \max_{a \in A} \Phi(a, b^*) = \Phi(a^*, b^*)\\
	%		& {} \textrm{(recall }\Phi = \Phi^A = - \Phi^B\textrm{)}
	%	\end{align*}
	%\end{frame}

	%\begin{frame}{\secname}
	%	\remark{DELETE IF TOO LONG}
	%	\begin{tcolorbox}[colback=rwth-10,colframe=rwth,title=value and saddle point, sharp corners]
	%		Let \eqref{eq-assumption-zerosum} hold. The zero-sum game has a value $V$ iff a saddle point $(a^*, b^*)$ exists. In this case
	%		$$V = V^- = V^+ = \Phi(a^*, b^*).$$
	%	\end{tcolorbox}
	%
	%	\vfill
	%	
	%	\textcolor{rwth}{Note:} As
	%	$$(a^*, b^*)\textrm{ Nash} \Leftrightarrow (a^*, b^*) \textrm{ saddle point} \Leftrightarrow V = \Phi(a^*, b^*)$$
	%	all Nash equilibria yield the same payoff.
	%	
	%	\vfill
	%	
	%	\textcolor{rwth}{Note:} value of the game well defined (at least for randomized strategies)
	%\end{frame}

	% note: this is the short version of previous two
	\begin{frame}{\secname}
		%\remark{delete if too SHORT}
		\begin{tcolorbox}[colback=rwth-10,colframe=rwth,title=value and Nash equilibrium, sharp corners]
			Let \eqref{eq-assumption-zerosum} hold. The zero-sum game has a value $V$ iff a Nash equilibrium $(a^*, b^*)$ exists. In this case
			$$V = V^- = V^+ = \Phi(a^*, b^*).$$
		\end{tcolorbox}
		
		\vfill
		
		\textcolor{rwth}{Note:} In Zero-Sum Games all Nash equilibria yield the same payoff.
		
		\vfill
		
		\textcolor{rwth}{Note:} The value of the game is well defined (at least for randomized strategies)
	\end{frame}

    \section{The Cooperative-Competitive Solution}
    \begin{frame}{\secname}
        Setting:
        \begin{itemize}
            \item 2 player static game (no zero-sum)
                $$\Phi^A, \Phi^B: A \times B \rightarrow \R$$
            \item goal: adopt $(a^\#, b^\#)$ such that
                $$V^\# \coloneqq \Phi^A(a^\#, b^\#) + \Phi^B(a^\#, b^\#) = \max_{(a,b) \in A \times B} \Phi^A(a, b) + \Phi^B(a, b)$$
            \item may need a side payment if e.g. $\Phi^B(a^\#, b^\#) \ll \Phi^A(a^\#, b^\#)$
        \end{itemize}
    \end{frame}

    \begin{frame}{\secname}
        Split the game
        $$\Phi^{\#}(a, b) \coloneqq \frac{\Phi^A(a, b) + \Phi^B(a, b)}{2}$$
        $$\Phi^{\flat}(a, b) \coloneqq \frac{\Phi^A(a, b) - \Phi^B(a, b)}{2}$$
        such that
        $$\Phi^A = \Phi^{\#} + \Phi^{\flat}$$
        $$\Phi^B = \Phi^{\#} - \Phi^{\flat}$$
        
        \vfill
        
        \textcolor{rwth}{Note:} $\Phi^\# \rightarrow \frac{V^\#}{2}$, $\Phi^{\flat} \rightarrow V^{\flat}$ (as value of the game)
    \end{frame}

    \begin{frame}{\secname}
		\begin{tcolorbox}[colback=rwth-10,colframe=rwth,title=co-co value, sharp corners]
            $$\left( \frac{V^{\#}}{2} + V^{\flat}, \frac{V^{\#}}{2} - V^{\flat} \right)$$
        \end{tcolorbox}
    
    	\vfill

		\begin{tcolorbox}[colback=rwth-10,colframe=rwth,title=co-co solution, sharp corners]
            strategies $(a^\#, b^\#) \in A \times B$, side payment $b \in \R$ from player B to player A such that
            $$\Phi^A(a^\#, b^\#) + p = \frac{V^{\#}}{2} + V^{\flat}$$
            $$\Phi^B(a^\#, b^\#) - p = \frac{V^{\#}}{2} - V^{\flat}$$
        \end{tcolorbox}
    \end{frame}

    \begin{frame}{\secname}
	\textcolor{rwth}{Example:} prisoner's dilemma
	
	\vfill
	
		\begin{figure}
			\centering
			\includegraphics[width=0.5\textwidth]{pics/prisoner_coco.jpg}
			\center{co-co solution in red; $b = 0$}
			
			\begin{minipage}{.5\textwidth}
				\centering
				\includegraphics[width=1.0\textwidth]{pics/prisoner_cococoop.jpg}
				\center{$\frac{V^\#}{2} = -1$}
			\end{minipage}%
			\begin{minipage}{.5\textwidth}
				\centering
				\includegraphics[width=1.0\textwidth]{pics/prisoner_cococomp.jpg}
				\center{$V^\flat = 0$}
			\end{minipage}
		\end{figure}
	\end{frame}

	\begin{frame}[plain]
		\centering
		\topskip0pt
		\vspace*{\fill}
		Thank you for your kind attention!
		\vspace*{\fill}
		%
	\end{frame}
\end{document}
