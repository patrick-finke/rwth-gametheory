\documentclass[11pt]{beamer}
\usepackage[utf8]{inputenc}
\usepackage[ngerman]{babel}
\usepackage{amsmath}
\usepackage{amsfonts}
\usepackage{amssymb}
\usepackage{tcolorbox}
\usepackage{mathtools} % \coloneqq
\usetheme{rwth}

% Define RWTH style colors

%%%%%%%%%%%%%%%%%%%%%%%%%%%%%%%%%%%%%%%%%%
% For HKS colors use the spotcolor package
% which predefines a number of such colors
% Main blue color is HKS 44 K
%%%%%%%%%%%%%%%%%%%%%%%%%%%%%%%%%%%%%%%%%%

%%%%%%%%%%%%%%%%%%%%%%%%%%%%%%%%%%%%%%%%%%
% RGB colors used for screen
%%%%%%%%%%%%%%%%%%%%%%%%%%%%%%%%%%%%%%%%%%
% The main blue color, 100 and 50 are the
% most commonly used ones
% e.g. in the logo
\xdefinecolor{rwth}   {RGB}{  0  84 159}
\xdefinecolor{rwth-75}{RGB}{ 64 127 183}
\xdefinecolor{rwth-50}{RGB}{142 186 229}
\xdefinecolor{rwth-25}{RGB}{199 221 242}
\xdefinecolor{rwth-10}{RGB}{232 241 250}

% All the other colors
% Secondary colors
\xdefinecolor{black}   {RGB}{  0   0   0}
\xdefinecolor{black-75}{RGB}{100 101 103}
\xdefinecolor{black-50}{RGB}{156 158 159}
\xdefinecolor{black-25}{RGB}{207 209 210}
\xdefinecolor{black-10}{RGB}{236 237 237}

\xdefinecolor{magenta}   {RGB}{227   0 102}
\xdefinecolor{magenta-75}{RGB}{233  96 136}
\xdefinecolor{magenta-50}{RGB}{241 158 177}
\xdefinecolor{magenta-25}{RGB}{249 210 218}
\xdefinecolor{magenta-10}{RGB}{253 238 240}

\xdefinecolor{yellow}   {RGB}{255 237   0}
\xdefinecolor{yellow-75}{RGB}{255 240  85}
\xdefinecolor{yellow-50}{RGB}{255 245 155}
\xdefinecolor{yellow-25}{RGB}{255 250 209}
\xdefinecolor{yellow-10}{RGB}{255 253 238}

% The extended color spectrum
\xdefinecolor{petrol}   {RGB}{  0  97 101}
\xdefinecolor{petrol-75}{RGB}{ 45 127 131}
\xdefinecolor{petrol-50}{RGB}{125 164 167}
\xdefinecolor{petrol-25}{RGB}{191 208 209}
\xdefinecolor{petrol-10}{RGB}{230 236 236}

\xdefinecolor{turkis}   {RGB}{  0 152 161}
\xdefinecolor{turkis-75}{RGB}{  0 177 183}
\xdefinecolor{turkis-50}{RGB}{137 204 207}
\xdefinecolor{turkis-25}{RGB}{202 231 231}
\xdefinecolor{turkis-10}{RGB}{235 246 246}

\xdefinecolor{grun}   {RGB}{ 87 171  39}
\xdefinecolor{grun-75}{RGB}{141 192  96}
\xdefinecolor{grun-50}{RGB}{184 214 152}
\xdefinecolor{grun-25}{RGB}{221 235 206}
\xdefinecolor{grun-10}{RGB}{242 247 236}

\xdefinecolor{maigrun}   {RGB}{189 205   0}
\xdefinecolor{maigrun-75}{RGB}{208 217  92}
\xdefinecolor{maigrun-50}{RGB}{224 230 154}
\xdefinecolor{maigrun-25}{RGB}{240 243 208}
\xdefinecolor{maigrun-10}{RGB}{249 250 237}

\xdefinecolor{orange}   {RGB}{246 168   0}
\xdefinecolor{orange-75}{RGB}{250 190  80}
\xdefinecolor{orange-50}{RGB}{253 212 143}
\xdefinecolor{orange-25}{RGB}{254 234 201}
\xdefinecolor{orange-10}{RGB}{255 247 234}

\xdefinecolor{rot}   {RGB}{204   7  30}
\xdefinecolor{rot-75}{RGB}{216  92  65}
\xdefinecolor{rot-50}{RGB}{230 150 121}
\xdefinecolor{rot-25}{RGB}{243 205 187}
\xdefinecolor{rot-10}{RGB}{250 235 227}

\xdefinecolor{bordeaux}   {RGB}{161  16  53}
\xdefinecolor{bordeaux-75}{RGB}{182  82  86}
\xdefinecolor{bordeaux-50}{RGB}{205 139 135}
\xdefinecolor{bordeaux-25}{RGB}{229 197 192}
\xdefinecolor{bordeaux-10}{RGB}{245 232 229}

\xdefinecolor{violett}   {RGB}{ 97  33  88}
\xdefinecolor{violett-75}{RGB}{131  78 117}
\xdefinecolor{violett-50}{RGB}{168 133 158}
\xdefinecolor{violett-25}{RGB}{210 192 205}
\xdefinecolor{violett-10}{RGB}{237 229 234}

\xdefinecolor{lila}   {RGB}{122 111 172}
\xdefinecolor{lila-75}{RGB}{155 145 193}
\xdefinecolor{lila-50}{RGB}{188 181 215}
\xdefinecolor{lila-25}{RGB}{222 218 235}
\xdefinecolor{lila-10}{RGB}{242 240 247}


\def\remark#1{\textcolor{red}{#1}}

\def\R{\mathbb{R}}

\AtBeginSection{\frame[noframenumbering,plain]{
		\frametitle{}
		\tableofcontents[currentsection]
}}

\begin{document}
    \title[Seminar Differential Games]{Seminar Differential Games}
    \author[P. Finke]{Patrick Finke}
    \email{patrick.finke@rwth-aachen.de}
	\subtitle{Introduction to Static Games}
	%\logo{}
	%\institute{}
    \date[]{October 17th, 2018}
	%\subject{}
	%\setbeamercovered{transparent}
	%\setbeamertemplate{navigation symbols}{}
	\frame[plain]{\maketitle}

    \begin{frame}
        \tableofcontents
    \end{frame}
	
    \section{Introduction}
    \begin{frame}{\secname}
		\begin{tcolorbox}[colback=rwth-10,colframe=rwth,title=Static Game for two players, sharp corners]
            \begin{itemize}
                \item two sets of strategies $A$ and $B$
                \item payoff functions $\Phi^A, \Phi^B: A \times B \rightarrow \R$
                \item goal of player A: $\max_{a \in A} \Phi^A(a, b)$
                \item goal of player B: $\max_{b \in B} \Phi^B(a, b)$
            \end{itemize}
        \end{tcolorbox}

        \vfill

		\begin{tcolorbox}[colback=rwth-10,colframe=rwth,title=Assumption, sharp corners]
            $A, B$ compact metric spaces, $\Phi^A, \Phi^B$ continuous
        \end{tcolorbox}
	\end{frame}

    \section{Example}

    \section{Solution Concepts}
    % Pareto
    \begin{frame}{\secname}
		\begin{tcolorbox}[colback=rwth-10,colframe=rwth,title=Pareto optimality, sharp corners]
            $(a^*, b^*)$ is Pareto optimal if there exist no $(a, b) \in A \times B$ such that
            $$\Phi^A(a, b) > \Phi^A(a^*, b^*) \quad \text{and} \quad \Phi^B(a, b) \geq \Phi^B(a^*, b^*)$$
            \center{or}
            $$\Phi^B(a, b) > \Phi^B(a^*, b^*) \quad \text{and} \quad \Phi^A(a, b) \geq \Phi^A(a^*, b^*)$$
        \end{tcolorbox}

        \vfill

        It is not possible to strictly increase the payoff of one player without strictly decreasing the payoff of the other.
    \end{frame}

    % Stackelberg
    \begin{frame}{\secname}
        set of best replies: $R^B(a) = \{ b \in B \mid \Phi^B(a,b) = \max_{\omega \in B} \Phi^B(a, \omega) \}$

        \vfill

        Idea:
        \begin{itemize}
            \item A leader, B follower
            \item A announces $a \in A$
            \item B chooses $b^* \in R^B(a)$ say $b^* = \beta(a)$
            \item goal of player A is $\max_{a \in A} \Phi^A(a, \beta(a))$
        \end{itemize}

        \vfill

		\begin{tcolorbox}[colback=rwth-10,colframe=rwth,title=Stackelberg equilibrium, sharp corners]
            $(a_S, b_S)$ is Stackelberg equilibrium if
            \begin{itemize}
                \item $b_S \in R^B(a_S)$
                \item $\Phi^A(a, b) \leq \Phi^A(a_S, b_S) \quad \forall (a, b) \textrm{ with } b_S \in R^B(a)$
            \end{itemize}
        \end{tcolorbox}

        \vfill

        {\color{red} definition in worten}
    \end{frame}

    \begin{frame}{\secname}
		\begin{tcolorbox}[colback=rwth-10,colframe=rwth,title=Nash equilibrium, sharp corners]
            $(a^*, b^*)$ is Nash equilibrium if for every $a \in A, b \in B$
            $$\Phi^A(a, b^*) \leq \Phi^A(a^*, b^*)$$
            \center{and}
            $$\Phi^B(a^*, b) \leq \Phi^B(a^*, b^*)$$
        \end{tcolorbox}

        \vfill

        No player can change his payoff by changing his strategy as long as the other player sticks to the equilibrium strategy.
    \end{frame}

    \section{Existence of Nash Equilibria}
    \begin{frame}{\secname}
		\begin{tcolorbox}[colback=rwth-10,colframe=rwth,title=Theorem, sharp corners]
            Assume $A, B \subset \R^n$ compact, convex; $\Phi^A, \Phi^B$ continuous and
            $$a \mapsto \Phi^A(a,b)\textrm{ is a concave function of }a\textrm{, for all }b \in B$$
            $$b \mapsto \Phi^B(a,b)\textrm{ is a concave function of }b\textrm{, for all }a \in A$$
            Then the non-cooperative game admits a Nash equilibrium.
        \end{tcolorbox}


		\begin{tcolorbox}[colback=rwth-10,colframe=rwth,title=Kakutani, sharp corners]
            Let $K \subset \R^n$ compact, convex. Let $F: K \mapsto \R^n$ be an upper semicontinuous multifunction with compact, convex values such that $F(x) \subset K$ for all $x \in K$. Then there exists $\bar{x} \in K$ such that
            $$\bar{x} \in F(\bar{x}).$$
        \end{tcolorbox}
    \end{frame}

    \section{Randomized Strategies}

    \section{Zero-Sum Games}

    \section{The Cooperative-Competitive Solution}
    \begin{frame}{\secname}
        Setting:
        \begin{itemize}
            \item 2 player game (no zero-sum)
                $$\Phi^A, \Phi^B: A \times B \rightarrow \R$$
            \item goal: adopt $(a^\#, b^\#)$ such that
                $$V^\# \coloneqq \Phi^A(a^\#, b^\#) + \Phi^B(a^\#, b^\#) = \max_{(a,b) \in A \times B} \Phi^A(a, b) + \Phi^B(a, b)$$
            \item may need a side payment if e.g. $\Phi^B(a^\#, b^\#) \ll \Phi^A(a^\#, b^\#)$
        \end{itemize}
    \end{frame}

    \begin{frame}{\secname}
        Split the game
        $$\Phi^{\textrm{coop}}(a, b) \coloneqq \frac{\Phi^A(a, b) + \Phi^B(a, b)}{2}$$
        $$\Phi^{\textrm{comp}}(a, b) \coloneqq \frac{\Phi^A(a, b) - \Phi^B(a, b)}{2}$$
        such that
        $$\Phi^A = \Phi^{\textrm{coop}} + \Phi^{\textrm{comp}}$$
        $$\Phi^B = \Phi^{\textrm{coop}} - \Phi^{\textrm{comp}}$$
    \end{frame}

    \begin{frame}{\secname}
		\begin{tcolorbox}[colback=rwth-10,colframe=rwth,title=co-co value, sharp corners]
            $$\left( \frac{V^{\textrm{coop}}}{2} + V^{\textrm{comp}}, \frac{V^{\textrm{coop}}}{2} - V^{\textrm{comp}} \right)$$
        \end{tcolorbox}

		\begin{tcolorbox}[colback=rwth-10,colframe=rwth,title=co-co solution, sharp corners]
            strategies $(a^\#, b^\#) \in Å \times B$, side payment $b$ from player B to player A such that
            $$\Phi^A(a^\#, b^\#) + p = \frac{V^{\textrm{coop}}}{2} + V^{\textrm{comp}}$$
            $$\Phi^B(a^\#, b^\#) - p = \frac{V^{\textrm{coop}}}{2} - V^{\textrm{comp}}$$
        \end{tcolorbox}
    \end{frame}

    \begin{frame}{\secname}
        \remark{Beispiel prisoners dilemma}
    \end{frame}
\end{document}
